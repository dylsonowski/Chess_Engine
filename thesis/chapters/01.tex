\chapter{Introduction}
The current electronic and IT industry is very minded on past and precis problem solutions. Very popular practice among commercial industries is processes automatization. A problems that cannot be solved by human is an increasingly common occurrence. Even if humans are very powerful entities, there are some biological limitations and it is impossible for human being to compete with machines in some situations. Additionally, using machines to solve some problems for humans is just faster and more practical approach. That is why usage of machine learning is becoming more and more popular. Usage of machine learning becomes so popular that this solution found usage, for example in process of recognizing speech and handwriting, personalization of advertisement and all web content, advanced security systems, antivirus and antimalware softwares, unmanned machines control algorithms and even in video games for creating NPCs (\textit{Non Playable Character} is an entity in video game with which player can interact but it is controlled by artificial intelligence). Those are only couple of example situation ion which artificial intelligence is used but there are many more possible usages of this solution. It is also possible to use machine learning algorithms in problem of playing chess.

In the 50s of the last century Alan Turing design experiment ,,The Imitation Game'' to prove that machine is capable of thinking. Turing defined set of rules that machine needs to evince to be called ,,intelligent''. This set of rules is still used today as an test called \textbf{Turing Test}. This test allows to specify if machine is capable of imitating human behavior. The most common execution of the Turing test is by making situation in which human interact witch machine and if participant cannot recognize that hes partner is a machine, test is considered as passed. 

Unfortunately, first approaches to making chess playing AI were unsuccessful. Chess playing softwares that were created in 40s century, turned out to be so demanding that existing machines couldn't provide enough computing power. This situation has improved in the 50s of the last century. Researchers from IBM company used improved Turing algorithm to perform a game of checkers in which machine was able to win. First application focussed on playing chess has been created by Dietrich Prinz in 1952. Unfortunately several years had to pass for efficient machines to appear. First machine fucuses mainly on chess playing was computer ,,Belle'', created in 1993. In May 1997, disruptive event happened. ,,Deep Blue'' computer, created by company IBM, achieved master level in chess, by wining against chess world champion Garri Kasparow. Final result of this game was 2:1 for Deep Blue (excluding three draws). After the success of Deep Blue, chess playing applications were constantly improving which result in troublefree overcoming human opponents. This also begin organizing chess tournaments in which only ,,virtual'' players were participating.

This master thesis describe problem of chess playing effectiveness based on used type of neural network. Final result of performed experiments will be conclusion about which of the used types of neural network performs better in given problem. Both of the neural network instances will be responsible for evaluating chessboard situation. Thesis text consists of four main chapters. First chapter (,,Problem analysis'') describes theoretical issues necessary for further experiments. Next chapter (,,Subject of the thesis'') described basic assumptions and tool uses for performing experiments. Chapter ,,Experiments'' describes methodology of testing, used data sets and presents gathered results. The last chapter (,,Summary'') consists of final conclusions and information about potential future research.

%\begin{itemize}
%\item introduction into the problem domain
%\item settling of the problem in the domain
%\item objective of the thesis 
%\item scope of the thesis
%\item short description of chapters
%\item clear description of contribution of the thesis's author
%\end{itemize}

