\subsubsection*{Thesis title}  
\Title

\subsubsection*{Abstract} 
This study analyze problem of machine playing chess. Following thesis describe all theoretical issues necessary for understanding the problem and it describe process of preparing testing environment. Document also describe methodology behind all experiments and testing process. Approach to the problem presented in this thesis, consider using neural network to evaluate situation on chessboard. The main question that needs to be answered is which type of neural network provide the best effectiveness in game of chess. Two types of neural networks that are analyze in this thesis are artificial neural network and convolutional neural network.

The most important topics that are described in the thesis are game trees and both used neural networks. To make sure that all acquired results are the most accurate, created AI instances are implemented from scratch, using C++ language, and tested on created application. Both of the neural networks has been tested in game against real opponent and against one another. To make gathered results even more accurate, both neural network instances was trained using the same data set and by the same number of training iterations. Thesis also includes description of training process and used data set which consists of chessboard arrangements translated from PGN files.

Last two chapters of the thesis describes acquired results from all experiments. The most interesting are results from playing phase of experiments but training process results are also documented. The last chapter consists of final conclusions about results as well as potential improvements and future research possibility.

\subsubsection*{Key words}  
C++, AI, chess, board games, game trees, PGN, neural network, convolutional neural network, min-max algorithm

\subsubsection*{Tytuł pracy}
\begin{otherlanguage}{polish}
\TitleAlt
\end{otherlanguage}

\subsubsection*{Streszczenie} 
\begin{otherlanguage}{polish}
Niniejsza praca rozpatruje problematykę maszyny grającej w szachy. W pracy zawarte zostały zagadnienia teoretyczne niezbędne do zrozumienia problemu oraz opisuje proces tworzenia środowiska testowego. Dokument opisuje także metodykę testowania oraz wykonane eksperymenty. Podejście do problemu, zaproponowane w poniższej pracy, zakłada wykorzystanie sieci neuronowej do oceny sytuacji na szachownicy. Dokument dostarcza rozpatruje pytanie jaki typ sieci neuronowej gwarantuje lepszą efektywność podczas gry w szachy. Typy użytych sieci to sztuczna sieć neuronowa i konwolucyjna sieć neuronowa.

Najważniejsze zagadnienia poruszane w pracy to tematyka drzew gry oraz obie użyte sieci neuronowe. Aby zwiększyć poprawność uzyskanych wyników, obie instancje sztucznej inteligencji zostały zaimplementowane od podstaw z wykorzystaniem języka C++ oraz zostały one przetestowane w specjalnie stworzonej aplikacji. Obie instancje sieci neuronowej zostały przetestowane w grze przeciwko żywemu graczowi oraz przeciwko sobie. Aby zwiększyć poprawność wyników jeszcze bardziej, obie sieci neuronowe zostały wytrenowane z użyciem tego samego zestawu danych. Praca zawiera także opis procesu uczenia oraz użyty zbiór danych który zawiera układy szachownicy pozyskane z plików PGN.

Dwa ostatnie rozdziały opisują uzyskane wyniki. Najbardziej interesujące są wyniki z fazy gry ale wyniki procesu uczenia również zostały udokumentowane. Ostatni rozdział opisuje ostateczne wnioski, informacje na temat możliwych usprawnień oraz potencjał badawczy w przyszłości.
\end{otherlanguage}

\subsubsection*{Słowa kluczowe} 
\begin{otherlanguage}{polish}
C++, sztuczna inteligencja, szachy, gry planszowe, drzewa gry, PGN, sieć neuronowa, konwolucyjna sieć neuronowa, algorytm minimaksowy
\end{otherlanguage}

